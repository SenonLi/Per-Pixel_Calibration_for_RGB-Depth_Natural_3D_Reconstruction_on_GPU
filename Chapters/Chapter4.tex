% Chapter 3
\chapter{Conclusion and Future Work} % Main chapter title
\label{sens_ConclusionAndFutureWork} 
%
RGB-D cameras have both of lens distortions and depth distortion problem. Distortions correction is wildly discussed in image processing area. In this thesis, a data-based LUT software calibration method for general RGB-D cameras is proposed, in contrast with the pinhole-camera-model based traditional method. In this section, both of the advantages and and disadvantages are concluded, and the possible avenues of future study are discussed.
%
\section{Conclusion}
In traditional 3D cameras' calibration, 3D reconstruction (based on pinhole camera model) and lens distortion are separated, which depresses the efficiency of image processing. Moreover, it does not include the depth distortion correction, assuming that the depth values are accurate and employing them directly as \(Z^w\) in 3D reconstruction.%
\\\\%
In stead of using a pinhole camera model, which in practical is not ideally accurate, our new proposed calibration method is based on real data. Not only \(X^w\)/\(Y^w\) values are rectified directly through one step transformation, but \(Z^w\) is also guaranteed accurate by importing externally measured data. 
\\\\%
During this data-based calibration method, high-order polynomial surface mapping is employed for \(X^w\)/\(Y^w\) rectification, as discussed in section \ref{sectionHighOrderPolynomialSurfaceMapping}. And an IoT technology, an individual BLE OF tracking module is introduced for the support of external \(Z^w\) data, offering a way for the depth distortion correction.
\\\\%
In contrast with the traditional pinhole-camera-model based method, the new proposed data-based method offers more efficient rectified 3D reconstruction coordinates, with rectified \(Z^w\) after the depth distortion correction. One disadvantage is the memory cost. The data-based method brings in better accuracy at the expense of more data collection. With the help of the BLE OF tracking module, data could be collected automatically during one-way of sliding the slider on the rail from one end to the other, whereas the memories cost cannot be skipped. However, with the fast development of semiconductor technologies, memories cost should not be a problem.

\section{Future Work}
%
The new proposed data-based method applies universally to all of the RGB-D cameras. With a better calibration system and corresponding DIP technologies, there could be a huge improvement space for calibration accuracy.
\\\\%
An apparent limit during the \(X^w\)/\(Y^w\) rectifications is the static patten, which offers only uncontrollable dead distortion information. As the working distance changes, the dot-clusters, which contains distortion information, observed by a camera also changes, resulting an inconsistent resolution of rectification. What's worse, for different RGB-D cameras with different resolution, the pattern (size) needs to be changed for a better match. 
\\\\%
In the future calibration system, in a lab with high-performance light control to reduce possible noises, the uniform pattern could be displayed by a large LCD screen. Managed thus, the uniform pattern is totally controllable, and will be part of the calibration system to form a new close-loop system. Data extracted from the camera stream supplying distortion information will be feedback for the best live pattern generating, helping adjust both size and distribution of the displayed pattern in real-time. In this way, we can control a fixed pattern distribution (offering uniformly distortion information) for cameras with various resolution at any working distance all along the rail.
\\\\%
The dots pattern could also be changed to uniform grid pattern (like a checker board) for more precise coordinates extraction. And corresponding DIP methods' improvement can also improve the accuracy for camera calibration.

%Chapter 4: results for 3 types of RGB-D cameras.\\
%Chapter 5: conclusion\\





































